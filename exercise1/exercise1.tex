\documentclass[12pt]{scrartcl}%{article} % Beginn der LaTeX-Datei

%% twocolumn

\usepackage{amsmath,amssymb}  % erleichtert Mathe 
\usepackage{enumerate}% schicke Nummerierung

\usepackage{graphicx} % für Grafik-Einbindung
%\usepackage{hyperref}

\usepackage[ngerman]{babel}
\usepackage[T1]{fontenc}
\usepackage{lmodern}
\usepackage[boxruled]{algorithm2e}
 % Einstellungen, wenn man deutsch schreiben will, z.B. Trennregeln
\usepackage[utf8]{inputenc}  % für Unix-Systeme
  % ermöglicht die direkte Eingabe von Umlauten und ß
  % evt. obige Zeile ersetzen durch
  % \usepackage[ansinew]{inputenc}  % für Windows
  % \usepackage[applemac]{inputenc} % für den Mac


%%%%%%%%%%%%%%%%%%%%%%%%%%%%%%%%%%%%%%%%%%%%%%%%%%%%%%%%%%%%%%%%%%
%
%  ntheorem
%
\usepackage[thmmarks,amsmath,hyperref,noconfig]{ntheorem} 
  % erlaubt es, Sätze, Definitionen etc. einfach durchzunummerieren.
\newtheorem{satz}{Satz}[section] % Nummerierung nach Abschnitten
\newtheorem{hilfssatz}[satz]{Hilfssatz}
\newtheorem{kor}[satz]{Korollar}

\theorembodyfont{\upshape}
\newtheorem{beispiel}[satz]{Beispiel}
\newtheorem{bemerkung}[satz]{Bemerkung}
\newtheorem{definition}[satz]{Definition} %[section]

\theoremstyle{nonumberplain}
\theoremheaderfont{\itshape}
\theorembodyfont{\normalfont}
\theoremseparator{.}
\theoremsymbol{\ensuremath{_\blacksquare}}
\newtheorem{beweis}{Beweis}
\qedsymbol{\ensuremath{_\blacksquare}}
%\theoremclass{LaTeX}
%
% Ende ntheorem
%
%%%%%%%%%%%%%%%%%%%%%%%%%%%%%%%%%%%%%%%%%%%%%%%%%%%%%%%%%%%%%%%%%%


%\pagestyle{empty}
%
% Ändern der bedruckten Fläche der Seite
% \addtolength{\textwidth}{3cm}  % Befehl mit zwei Argumenten
% \addtolength{\textheight}{3cm}
% \hoffset-2cm % verschiebt das Textfenster nach links
% \voffset-5mm % verschiebt das Textfenster nach oben
%
%\parindent=0pt %% keine Einzug zu Beginn von Abs\"atzen
%\parskip=2mm   %% erzeugt einen zusätzliche Zeilenabstand zwischen
                %% Absätzen. Nötig bei \parindent=0pt


%%%%%%%%%%%%%%%%%%%%%%%%%%%%%%%%%%%%%%%%%%%%%%%%%%%%%%%%%%%%%%%%%%
% ermöglicht, farbigen Text zu drucken.
\usepackage{color}
% Und nun werden die Farben definiert - daran können Sie nach Belieben selber rumspielen.
\definecolor{white}{rgb}{1,1,1}
\definecolor{darkred}{rgb}{0.3,0,0}
\definecolor{darkgreen}{rgb}{0,0.3,0}
\definecolor{darkblue}{rgb}{0,0,0.3}
\definecolor{pink}{rgb}{0.78,0.09,0.51}
\definecolor{purple}{rgb}{0.28,0.24,0.55}
\definecolor{orange}{rgb}{1,0.6,0.0}
\definecolor{grey}{rgb}{0.4,0.4,0.4}
\definecolor{aquamarine}{rgb}{0.4,0.8,0.65}


\DeclareMathOperator{\GL}{GL} % einige Macro, siehe am Ende Abschn. 2
\newcommand{\N}{\mathbb{N}}
\newcommand{\Z}{\mathbb{Z}}
\newcommand{\Q}{\mathbb{Q}}
\newcommand{\R}{\mathbb{R}}
\newcommand{\C}{\mathbb{C}}
\newcommand{\cP}{{\mathcal P}} 
\newcommand{\bO}[1]{\mathcal O(#1)}

\begin{document}

\author{Dennis Hempfing, Sebastian Koall}
\title{Übung 1}
\date{} 
\pagestyle{myheadings}
\markright{\hfill Dennis Hempfing, Sebastian Koall}

\maketitle % erzeugt den Kopf
 
\section*{Aufgabe 2}
\subsection*{a)}

\begin{align*}
4 &= \log_{10}{(7x+51)} + \log_{10}{(15x-5)} &&|\text{ Logarithmusgesetz anwenden} \\
4 &= \log_{10}{((7x+51) * (15x-5))} &&|\text{ Logarithmus in Potenz umschreiben} \\
4^{10} &= (7x+51) * (15x-5) \\
10000 &= 105x^2 - 35x + 765x - 255 &&|:5\\
2000 &= 21x^2 + 146x - 51 &&|:21\\
\frac{2000}{21} &= x^2 + \frac{146}{21} x - \frac{51}{21} &&|-\frac{2000}{21}\\
0 &= x^2 + \frac{146}{21} x - \frac{2051}{21} \\
x_{1/2} &= -\frac{146}{42} \pm \sqrt{\left(\frac{146}{42}\right)^2 + \frac{2051}{21}} \\
x_{1} &= 7 \\
x_{2} &= -\frac{293}{21}
\end{align*}

Die Lösung $x_2$ entfällt, da der Numerus für $\log_{10}{(7x+51)}$ gleich $-\frac{140}{3}$ und somit negativ wäre.

Die einzige mögliche Lösung für x ist demzufolge $x_1 = 7$.

\subsection*{b)}
\begin{align*}
\log_{2}(4x+4) &= \log_{2}{(2x-2)} + \log_{2}(x+1) &&|\text{ Logarithmus in Potenz umschreiben} \\
4x + 4 &= 2^{\log_{2}{(2x-2)} + \log_{2}(x+1)} &&|\text{ Logarithmusgesetz anwenden} \\
4x + 4 &= 2^{\log_{2}{(2x-2)}} * 2^{\log_{2}(x+1)} \\
4x + 4 &= (2x - 2) * (x + 1) \\
4x + 4 &= 2x^2 - 2 &&|-(4x+4)\\
0 &= 2x^2 - 4x - 6 &&|:2\\
0 &= x^2 - 2x - 3 \\
x_{1/2} &= \frac{2}{2} \pm \sqrt{\left(\frac{2}{2}\right)^2 + 3} \\
x_{1/2} &= 1 \pm \sqrt{4} \\
x_{1} &= 3 \\
x_{2} &= -1
\end{align*}

Die Lösung $x_2$ entfällt, da der Numerus für $\log_{2}{(2x-2)}$ gleich $-4$ und somit negativ wäre.

Die einzige mögliche Lösung für x ist demzufolge $x_1 = 3$.

\newpage

\section*{Aufgabe 3}
\subsection*{a)}

\subsection*{b)}

Zu beweisen ist

$ 2^{f(n)} = \mathcal{O}(2^{g(n)}) \rightarrow f(n) = \mathcal{O}(g(n)) $

Diese Aussage ist äquivalent zu

$ \exists c,  $

\newpage

\section*{Aufgabe 4}

\paragraph{Eingabe:} Array aus ganzen Zahlen $\mathbb{Z}$ der Größe n

\paragraph{Ausgabe:} ganze Zahl

\begin{algorithm}[H]
	
	smallest = 0\;
	secondSmallest = 0\;
	greatest = 0\;
	secondGreatest = 0\;
	
	temp = 0\;
	
	\For{number in listOfNumbers}{
		\If{number > 0}{\
			\If{number > greatest}{
				temp = greatest\;
				greatest = number\;
				secondGreatest = temp\;	
			}\Else(number > secondGreatest){
			secondGreatest = number\;
			}		
		}\ElseIf{number < 0}{
			\If{number < smallest}{
				temp = smallest\;
				smallest = number\;
				secondSmallest= temp\;				
			}\Else(elif number < secondSmallest){
				secondSmallest = number\;		
			}
		}
	}
	\If{greatest*secondGreatest > smallest*secondSmallest}{
		return greatest*secondGreatest\;
	}\Else{
		return smallest*secondSmallest\;
	}

\end{algorithm}

\vspace{0.3cm}

Der Algorithmus iteriert einmal über das Array der Größe n, die Laufzeit ist also von n abhängig. Pro Iteration werden 14 Schritte ausgeführt. Vor und nach den Iterationen werden insgesamt 9 Schritte ausgeführt. Demzufolge beträgt die Laufzeit $\mathcal{O}(14n+9) = \mathcal{O}(n)$.

\end{document}
