\documentclass[12pt]{scrartcl}%{article} % Beginn der LaTeX-Datei

%% twocolumn

\usepackage{amsmath,amssymb}  % erleichtert Mathe 
\usepackage{enumerate}% schicke Nummerierung

\usepackage{graphicx} % für Grafik-Einbindung
%\usepackage{hyperref}

\usepackage[ngerman]{babel}
\usepackage[T1]{fontenc}
\usepackage{lmodern}
\usepackage[boxruled]{algorithm2e}
\usepackage{parskip}
 % Einstellungen, wenn man deutsch schreiben will, z.B. Trennregeln
\usepackage[utf8]{inputenc}  % für Unix-Systeme
  % ermöglicht die direkte Eingabe von Umlauten und ß
  % evt. obige Zeile ersetzen durch
  % \usepackage[ansinew]{inputenc}  % für Windows
  % \usepackage[applemac]{inputenc} % für den Mac
\usepackage{listings}

%%%%%%%%%%%%%%%%%%%%%%%%%%%%%%%%%%%%%%%%%%%%%%%%%%%%%%%%%%%%%%%%%%
%
%  ntheorem
%
\usepackage[thmmarks,amsmath,hyperref,noconfig]{ntheorem} 
  % erlaubt es, Sätze, Definitionen etc. einfach durchzunummerieren.
\newtheorem{satz}{Satz}[section] % Nummerierung nach Abschnitten
\newtheorem{hilfssatz}[satz]{Hilfssatz}
\newtheorem{kor}[satz]{Korollar}

\theorembodyfont{\upshape}
\newtheorem{beispiel}[satz]{Beispiel}
\newtheorem{bemerkung}[satz]{Bemerkung}
\newtheorem{definition}[satz]{Definition} %[section]

\theoremstyle{nonumberplain}
\theoremheaderfont{\itshape}
\theorembodyfont{\normalfont}
\theoremseparator{.}
\theoremsymbol{\ensuremath{_\blacksquare}}
\newtheorem{beweis}{Beweis}
\qedsymbol{\ensuremath{_\blacksquare}}
%\theoremclass{LaTeX}
%
% Ende ntheorem
%
%%%%%%%%%%%%%%%%%%%%%%%%%%%%%%%%%%%%%%%%%%%%%%%%%%%%%%%%%%%%%%%%%%


%\pagestyle{empty}
%
% Ändern der bedruckten Fläche der Seite
% \addtolength{\textwidth}{3cm}  % Befehl mit zwei Argumenten
% \addtolength{\textheight}{3cm}
% \hoffset-2cm % verschiebt das Textfenster nach links
% \voffset-5mm % verschiebt das Textfenster nach oben
%
\parindent=0pt %% keine Einzug zu Beginn von Abs\"atzen
\parskip=0.5mm   %% erzeugt einen zusätzliche Zeilenabstand zwischen
                %% Absätzen. Nötig bei \parindent=0pt


%%%%%%%%%%%%%%%%%%%%%%%%%%%%%%%%%%%%%%%%%%%%%%%%%%%%%%%%%%%%%%%%%%
% ermöglicht, farbigen Text zu drucken.
\usepackage{color}
% Und nun werden die Farben definiert - daran können Sie nach Belieben selber rumspielen.
\definecolor{white}{rgb}{1,1,1}
\definecolor{darkred}{rgb}{0.3,0,0}
\definecolor{darkgreen}{rgb}{0,0.3,0}
\definecolor{darkblue}{rgb}{0,0,0.3}
\definecolor{pink}{rgb}{0.78,0.09,0.51}
\definecolor{purple}{rgb}{0.28,0.24,0.55}
\definecolor{orange}{rgb}{1,0.6,0.0}
\definecolor{grey}{rgb}{0.4,0.4,0.4}
\definecolor{aquamarine}{rgb}{0.4,0.8,0.65}


\DeclareMathOperator{\GL}{GL} % einige Macro, siehe am Ende Abschn. 2
\newcommand{\N}{\mathbb{N}}
\newcommand{\Z}{\mathbb{Z}}
\newcommand{\Q}{\mathbb{Q}}
\newcommand{\R}{\mathbb{R}}
\newcommand{\C}{\mathbb{C}}
\newcommand{\cP}{{\mathcal P}}
\newcommand{\bO}[1]{\mathcal O(#1)}
\newcommand{\bT}[1]{\Theta (#1)}
\newcommand{\bOg}[1]{\Omega (#1)}

% Code-Macro, wenn man mal etwas als Code highlighten möchte
\newcommand{\code}[1]{\lstinline[basicstyle=\ttfamily\color{black}]{#1}}

\begin{document}

\author{Dennis Hempfing, Sebastian Koall}
\title{Übung 11}
\date{} 
\pagestyle{myheadings}
\markright{\hfill Dennis Hempfing, Sebastian Koall}

\maketitle % erzeugt den Kopf

\section*{Aufgabe 2}

\subsection*{Teilaufgabe a)}

\paragraph{Anname:} Beim Finden einer Lösung für dieses Problem werden Kanten zwischen zwei Knoten gelöscht, wenn ein kürzer Pfad hinzugefügt wird, sodass die Kantenmenge am Ende nur die Kanten beinhaltet, die das optimale Streckennetz ergeben. (In der Realität zum Beispiel umgesetzt durch Stilllegungen.)
\paragraph{Eingabe:} ungerichteter, zusammenhängender Graph G=(V,E), \\Gewichtsfunktion w: E $\rightarrow \Q ^+$
\paragraph{Ausgabe:} ungerichteter, zusammenhängender Graph G'=(V',E')
\paragraph{Korrektheit:} $|V'| \ge |V| \wedge \forall G''=(V'', E''): |V''| \ge |V|: \sum_{e' \in E'}^{}w(e') \le \sum_{e'' \in E''}^{}w(e'') \wedge \sum_{e' \in E'}^{}w(e') < \sum_{e \in E}^{}w(e)$
\paragraph{Interpretation:} G ist das bisher vorhandene Streckennetz zwischen den einzelnen Städten. w ist die Gewichtungsfunktion, die jeder Kante aus E ein Gewicht aus den positiven Rationalen Zahlen zuweist. G' ist der errechnete Graph. G' ist eine valide Lösung, wenn die Knotenmenge V' aus G' mindestens genauso viele Knoten wie V aus G enthält und wenn die Summe der Kantengewichte der Kanten aus E' kleiner gleich der Summe der Kantengewichte der Kanten aus E ist.

\subsection*{Teilaufgabe b)}

\paragraph{Eingabe:} ungerichteter, zusammenhängender Graph G(V,E), Gewichtsfunktion w: E $\rightarrow \Q ^+$
\paragraph{Ausgabe:} $v \in V$
\paragraph{Korrektheit:} $\exists v \forall v' \in V: v \not= v' \wedge \exists Pfad \ p = v,...,v': \forall Pfade \ p' = v,...,v':  \sum_{e' \in p'}^{}w(e) \ge \sum_{e \in p}^{}w(e) \wedge \forall v' \in V: v' \not= v : \sum_{p_v  \in P_v}{w(p_v)} < \sum_{p_{v'} \in P_{v'}}{w(p_{v'})}$

\paragraph{Interpretation:}
Der Knoten $v$ ist Hauptstadt und die Pfade von $v$ zu einem beliebigen von $v$ verschiedenen Knoten haben stets die minimal mögliche Länge. Außerdem gibt keine Knoten $v'$, für den gilt: Die Summe aller Kantengewichte auf den verschiedenen Pfaden ist kleiner als die Summe aller Kantengewichte der Pfade von $v$.

\subsection*{Teilaufgabe c)}

\paragraph{Eingabe:} Menge $S \subset \{A-Z\}^*$
\paragraph{Ausgabe:} $w \in \{A-Z\}^*$
\paragraph{Korrektheit:} $\exists w \in \{A-Z\}^*: \forall a \in S \ \exists v,x \in \{A-Z\}^*: a = vwx \wedge |v| \ge 0, |x| \ge 0, |w| \ge 1: \forall w' \in \{A-Z\}^*: a = vw'x \wedge  |w'| \le |w|$
\paragraph{Interpretation:}
In einer Menge von Eingabestrings $S$ soll die längste gemeinsame Sequenz $w$ an Buchstaben gesucht werden. Es existiert also ein String $w$, welcher sich in allen Wörtern $a$ aus $S$ befindet. Gleichzeitig gilt: Gibt es eine Wortsequenz $w'$, welche in allen Wörtern vorkommt, so ist $w$ stets länger als $w'$.

\subsection*{Teilaufgabe d)}

\paragraph{Eingabe:}
\paragraph{Ausgabe:} Personenzahl $P \in \N^+$
\paragraph{Korrektheit:}
\paragraph{Interpretation:}

\end{document}
