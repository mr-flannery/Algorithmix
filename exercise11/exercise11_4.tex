\documentclass[12pt]{scrartcl}%{article} % Beginn der LaTeX-Datei

%% twocolumn

\usepackage{amsmath,amssymb}  % erleichtert Mathe 
\usepackage{enumerate}% schicke Nummerierung

\usepackage{graphicx} % für Grafik-Einbindung
%\usepackage{hyperref}

\usepackage[ngerman]{babel}
\usepackage[T1]{fontenc}
\usepackage{lmodern}
\usepackage[boxruled]{algorithm2e}
\usepackage{parskip}
 % Einstellungen, wenn man deutsch schreiben will, z.B. Trennregeln
\usepackage[utf8]{inputenc}  % für Unix-Systeme
  % ermöglicht die direkte Eingabe von Umlauten und ß
  % evt. obige Zeile ersetzen durch
  % \usepackage[ansinew]{inputenc}  % für Windows
  % \usepackage[applemac]{inputenc} % für den Mac
\usepackage{listings}

%%%%%%%%%%%%%%%%%%%%%%%%%%%%%%%%%%%%%%%%%%%%%%%%%%%%%%%%%%%%%%%%%%
%
%  ntheorem
%
\usepackage[thmmarks,amsmath,hyperref,noconfig]{ntheorem} 
  % erlaubt es, Sätze, Definitionen etc. einfach durchzunummerieren.
\newtheorem{satz}{Satz}[section] % Nummerierung nach Abschnitten
\newtheorem{hilfssatz}[satz]{Hilfssatz}
\newtheorem{kor}[satz]{Korollar}

\theorembodyfont{\upshape}
\newtheorem{beispiel}[satz]{Beispiel}
\newtheorem{bemerkung}[satz]{Bemerkung}
\newtheorem{definition}[satz]{Definition} %[section]

\theoremstyle{nonumberplain}
\theoremheaderfont{\itshape}
\theorembodyfont{\normalfont}
\theoremseparator{.}
\theoremsymbol{\ensuremath{_\blacksquare}}
\newtheorem{beweis}{Beweis}
\qedsymbol{\ensuremath{_\blacksquare}}
%\theoremclass{LaTeX}
%
% Ende ntheorem
%
%%%%%%%%%%%%%%%%%%%%%%%%%%%%%%%%%%%%%%%%%%%%%%%%%%%%%%%%%%%%%%%%%%


%\pagestyle{empty}
%
% Ändern der bedruckten Fläche der Seite
% \addtolength{\textwidth}{3cm}  % Befehl mit zwei Argumenten
% \addtolength{\textheight}{3cm}
% \hoffset-2cm % verschiebt das Textfenster nach links
% \voffset-5mm % verschiebt das Textfenster nach oben
%
\parindent=0pt %% keine Einzug zu Beginn von Abs\"atzen
\parskip=0.5mm   %% erzeugt einen zusätzliche Zeilenabstand zwischen
                %% Absätzen. Nötig bei \parindent=0pt


%%%%%%%%%%%%%%%%%%%%%%%%%%%%%%%%%%%%%%%%%%%%%%%%%%%%%%%%%%%%%%%%%%
% ermöglicht, farbigen Text zu drucken.
\usepackage{color}
% Und nun werden die Farben definiert - daran können Sie nach Belieben selber rumspielen.
\definecolor{white}{rgb}{1,1,1}
\definecolor{darkred}{rgb}{0.3,0,0}
\definecolor{darkgreen}{rgb}{0,0.3,0}
\definecolor{darkblue}{rgb}{0,0,0.3}
\definecolor{pink}{rgb}{0.78,0.09,0.51}
\definecolor{purple}{rgb}{0.28,0.24,0.55}
\definecolor{orange}{rgb}{1,0.6,0.0}
\definecolor{grey}{rgb}{0.4,0.4,0.4}
\definecolor{aquamarine}{rgb}{0.4,0.8,0.65}


\DeclareMathOperator{\GL}{GL} % einige Macro, siehe am Ende Abschn. 2
\newcommand{\N}{\mathbb{N}}
\newcommand{\Z}{\mathbb{Z}}
\newcommand{\Q}{\mathbb{Q}}
\newcommand{\R}{\mathbb{R}}
\newcommand{\C}{\mathbb{C}}
\newcommand{\cP}{{\mathcal P}}
\newcommand{\bO}[1]{\mathcal O(#1)}
\newcommand{\bT}[1]{\Theta (#1)}
\newcommand{\bOg}[1]{\Omega (#1)}

% Code-Macro, wenn man mal etwas als Code highlighten möchte
\newcommand{\code}[1]{\lstinline[basicstyle=\ttfamily\color{black}]{#1}}

\begin{document}

\author{Dennis Hempfing, Sebastian Koall}
\title{Übung 11}
\date{} 
\pagestyle{myheadings}
\markright{\hfill Dennis Hempfing, Sebastian Koall}

\maketitle % erzeugt den Kopf

\section*{Aufgabe 4}

\paragraph{Idee:} Versucht man eine Zahl des Arrays a in eine Zahl des Arrays b umzuwandeln, gibt es zwei Möglichkeiten: entweder man löscht die Zahl (Kosten 3), oder man modifiziert sie (Kosten 2 für Verringern/4 für Erhöhen). Somit lässt sich ein Entscheidungsgraph bauen. Man betrachtet die erste Zahl aus a und die erste aus b und vergleicht sie miteinander. Das ist quasi der Startknoten. Dieser lässt sich mit zwei neuen Knoten verbinden, in denen man jeweils den neuen Zustand von a speichert sowie die Summe aller bisherigen Kosten. Wann immer sich zwei Pfade in einem Knoten treffen, speichert man sich die akkumulierten Kosten des günstigeren Pfads. Führt man dies bis zu Ende aus erhält man im Knoten, in dem a erfolgreich in b umgewandelt worden ist, die minimalen Kosten für diese Umwandlung, also das Ergebnis.

\paragraph{Korrektheit:} Wie bereits erläutert gibt es in jedem Zustand genau zwei Möglichkeit: Löschen oder Modifizieren. Mit diesem Algorithmus wird in jedem Zustand von a jede Möglichkeit samt ihrer Kosten berechnet und gespeichert. Jedes mal, wenn zwei Pfade aufeinander treffen, werden die günstigeren Kosten gespeichert. Es werden also alle Kosten für alle Möglichkeiten errechnet. Somit müssen die Kosten im finalen Knoten die minimalen Kosten für die Umwandlung von a in b sein.

\paragraph{Algorithmus:} Leider haben wir es nicht mehr geschafft Pseudocode zu schreiben, deswegen hier die Lösung als Python-Script.

Eingabe: Liste von Zahlen a der Länge n, Liste von Zahlen b der Länge m

Ausgabe: Umwandlungskosten $\in \N$

b = [1,2,3] // Zielarray

a = [4,3,2,1] // Startarray

delLimit = len(a) - len(b) + 1

//init Matrix

m = [[0 for x in range(delLimit)] for x in range(len(a))]

for i in range (0, delLimit):

	m[0][i] = i * 3

// first row

for i in range(1, len(a)):

	if a[i-1] > b[i-1]:
	
		m[i][0] = m[i-1][0] + 2;
		
	if a[i-1] < b[i-1]:
	
		m[i][0] = m[i-1][0] + 4;
		
	if a[i-1] == b[i-1]:
	
		m[i][0] = m[i-1][0];

// compute values

for j in range(1, delLimit):

	for i in range(1, len(a)):
	
		if a[i-1+j] > b[i-1]:
		
			m[i][j] = min(m[i-1][j] + 2, m[i][j-1] + 3)
			
		if a[i-1+j] < b[i-1]:
		
			m[i][j] = min(m[i-1][j] + 4, m[i][j-1] + 3)
			
		if a[i-1+j] == b[i-1]:
		
			m[i][j] = min(m[i-1][j], m[i][j-1] + 3)

output = ""

for i in range(0,len(a)):

	for j in range(0,delLimit):
	
		output += str(m[i][j]) + " "
		
	print output
	
	output = ""
	

\paragraph{Laufzeit:} Die Berechnung ist hier mithilfe einer Matrix der Größe $n * (n-m+1)$ realisiert. Im Verlaufe des Programms wird jede Zelle der Matrix mindestens einmal angefasst. Die Laufzeit liegt also bei $\bO{n^2}$. Die tatsächliche Laufzeit ist stark abhängig vom Betrag der Differenz zwischen $n$ und $m$, da sich dies unmittelbar auf die Größe der Matrix auswirkt.

\end{document}
