\documentclass[12pt]{scrartcl}%{article} % Beginn der LaTeX-Datei

%% twocolumn

\usepackage{amsmath,amssymb}  % erleichtert Mathe 
\usepackage{enumerate}% schicke Nummerierung

\usepackage{graphicx} % für Grafik-Einbindung
%\usepackage{hyperref}

\usepackage[ngerman]{babel}
\usepackage[T1]{fontenc}
\usepackage{lmodern}
\usepackage[boxruled]{algorithm2e}
\usepackage{parskip}
 % Einstellungen, wenn man deutsch schreiben will, z.B. Trennregeln
\usepackage[utf8]{inputenc}  % für Unix-Systeme
  % ermöglicht die direkte Eingabe von Umlauten und ß
  % evt. obige Zeile ersetzen durch
  % \usepackage[ansinew]{inputenc}  % für Windows
  % \usepackage[applemac]{inputenc} % für den Mac


%%%%%%%%%%%%%%%%%%%%%%%%%%%%%%%%%%%%%%%%%%%%%%%%%%%%%%%%%%%%%%%%%%
%
%  ntheorem
%
\usepackage[thmmarks,amsmath,hyperref,noconfig]{ntheorem} 
  % erlaubt es, Sätze, Definitionen etc. einfach durchzunummerieren.
\newtheorem{satz}{Satz}[section] % Nummerierung nach Abschnitten
\newtheorem{hilfssatz}[satz]{Hilfssatz}
\newtheorem{kor}[satz]{Korollar}

\theorembodyfont{\upshape}
\newtheorem{beispiel}[satz]{Beispiel}
\newtheorem{bemerkung}[satz]{Bemerkung}
\newtheorem{definition}[satz]{Definition} %[section]

\theoremstyle{nonumberplain}
\theoremheaderfont{\itshape}
\theorembodyfont{\normalfont}
\theoremseparator{.}
\theoremsymbol{\ensuremath{_\blacksquare}}
\newtheorem{beweis}{Beweis}
\qedsymbol{\ensuremath{_\blacksquare}}
%\theoremclass{LaTeX}
%
% Ende ntheorem
%
%%%%%%%%%%%%%%%%%%%%%%%%%%%%%%%%%%%%%%%%%%%%%%%%%%%%%%%%%%%%%%%%%%


%\pagestyle{empty}
%
% Ändern der bedruckten Fläche der Seite
% \addtolength{\textwidth}{3cm}  % Befehl mit zwei Argumenten
% \addtolength{\textheight}{3cm}
% \hoffset-2cm % verschiebt das Textfenster nach links
% \voffset-5mm % verschiebt das Textfenster nach oben
%
\parindent=0pt %% keine Einzug zu Beginn von Abs\"atzen
%\parskip=2mm   %% erzeugt einen zusätzliche Zeilenabstand zwischen
                %% Absätzen. Nötig bei \parindent=0pt


%%%%%%%%%%%%%%%%%%%%%%%%%%%%%%%%%%%%%%%%%%%%%%%%%%%%%%%%%%%%%%%%%%
% ermöglicht, farbigen Text zu drucken.
\usepackage{color}
% Und nun werden die Farben definiert - daran können Sie nach Belieben selber rumspielen.
\definecolor{white}{rgb}{1,1,1}
\definecolor{darkred}{rgb}{0.3,0,0}
\definecolor{darkgreen}{rgb}{0,0.3,0}
\definecolor{darkblue}{rgb}{0,0,0.3}
\definecolor{pink}{rgb}{0.78,0.09,0.51}
\definecolor{purple}{rgb}{0.28,0.24,0.55}
\definecolor{orange}{rgb}{1,0.6,0.0}
\definecolor{grey}{rgb}{0.4,0.4,0.4}
\definecolor{aquamarine}{rgb}{0.4,0.8,0.65}


\DeclareMathOperator{\GL}{GL} % einige Macro, siehe am Ende Abschn. 2
\newcommand{\N}{\mathbb{N}}
\newcommand{\Z}{\mathbb{Z}}
\newcommand{\Q}{\mathbb{Q}}
\newcommand{\R}{\mathbb{R}}
\newcommand{\C}{\mathbb{C}}
\newcommand{\cP}{{\mathcal P}}
\newcommand{\bO}[1]{\mathcal O(#1)}
\newcommand{\bT}[1]{\Theta (#1)}
\newcommand{\bOg}[1]{\Omega (#1)}

\begin{document}

\author{Dennis Hempfing, Sebastian Koall}
\title{Übung}
\date{} 
\pagestyle{myheadings}
\markright{\hfill Dennis Hempfing, Sebastian Koall}

\maketitle % erzeugt den Kopf

\section*{Aufgabe 1}

\subsection*{Teilaufgabe a)}
Rekurrenz: $ T(n) = 2T(n/2) + 5n $

\paragraph{obere Schranke:} $ T(n) \in \bT{n*\log(n)} $

\paragraph{Begründung:} Master-Theorem Fall 2

$f(n) \in \bT{n^{\log_b (a)}}$

$a=2, b=2, f(n)=5n$

$log_2 (2) = 1$

$5n \in \bT{n^{\log_b (a)}}$

$5n \in \bT{n^1}$ wahre Aussage

\newpage

\subsection*{Teilaufgabe b)}
Rekurrenz: $ T(n) = T(n/2) + 17 $

\paragraph{obere Schranke:} $ T(n) \in \bT{\log(n)} $

\paragraph{Begründung:} Master-Theorem Fall 2

$f(n) \in \bT{n^{\log_b (a)}}$

$a=1, b=2, f(n)=17$

$\log_2 (1) = 0$

$17 \in \bT{n^0}$

$17 \in \bT{1}$ wahre Aussage

\subsection*{Teilaufgabe c)}
Rekurrenz: $ T(n) = 3T(n/3) + n^3 $

\paragraph{obere Schranke:} $ T(n) \in \bT{n^3} $

\paragraph{Begründung:} Master-Theorem Fall 3

$f(n) \in \bOg{n^{\log_b (a) + \epsilon}} \text{ mit } \epsilon > 0 \text{ und } af(\frac{n}{b}) \le cf(n) \text{ mit } 0 < c < 1$ 

$a=3, b=3, f(n)=n^3$

$log_3 (3) = 1$

$n^3 \in \bOg{n^{1+2}} \text{ mit } \epsilon = 2$

$n^3 \in \bOg{n^3}$  wahre Aussage

$3*f(\frac{n}{3}) \le c * f(n)$

$3*\left( \frac{n}{3} \right) ^2 \le c * n^3$

$\frac{1}{3} n^3 \le c * n^3 \text{ mit } c=\frac{1}{3}$ wahre Aussage

\newpage

\subsection*{Teilaufgabe d)}
Rekurrenz: $ T(n) = 5T(n/8) + n*\log(n) $

\paragraph{obere Schranke:} $ T(n) \in \bT{n*\log(n)} $

\paragraph{Begründung:} Master-Theorem Fall 3

$f(n) \in \bOg{n^{\log_b (a) + \epsilon}} \text{ mit } \epsilon > 0 \text{ und } af(\frac{n}{b}) \le cf(n) \text{ mit } 0 < c < 1$

$a=5, b=8, f(n)=n*\log (n)$

$0 < log_8 (5) < 1$

% ACHTUNG: Bitte hier schaun ob das halbwegs sinn macht, weil eigentlich muss das ja n log(n) sein
$n * log (n) \in \bOg{n^{log_8 (5) + \epsilon}} \text{ mit } \epsilon \approx 0,23 \text{ (Laut Cormen geht das, siehe Edition 3 Seite 95)}$  wahre Aussage

$5*\frac{n}{8} * \log{\frac{n}{8}} \le c * n * log(n)$

$\frac{5}{8} * \log{\frac{n}{8}} \le c * log(n) \text{ mit } c=\frac{5}{8}$ wahre Aussage

\subsection*{Teilaufgabe e)}
Rekurrenz: $ T(n) = 7T(n/2) + n^2 $

\paragraph{obere Schranke:} $ T(n) \in \bT{n^{\log_2(7)}} $

\paragraph{Begründung:} Master-Theorem Fall 1

$f(n) \in \bOg{n^{\log_b (a) - \epsilon}} \text{ mit } \epsilon > 0 $

$a=7,b=2, f(n) = n^2$

$n^2 \in \bO{n^{\log_2 (7) - \epsilon}}$

$n^2 \in \bO{n^2} \text{ mit } \epsilon \approx 0,8$ wahre Aussage

\newpage

\subsection*{Teilaufgabe f)}
Rekurrenz: $ T(n) = 27T(n/3) + 1 $

\paragraph{obere Schranke:} $ T(n) \in \bT{n^3} $

\paragraph{Begründung:} Master-Theorem Fall 1

$f(n) \in \bOg{n^{\log_b (a) - \epsilon}} \text{ mit } \epsilon > 0 $

$a=27, b=3, f(n) = 1$

$\log_3 (27) = 3$

$1 \in \bO{n^3-\epsilon}$

$1 \in \bO{1} \text{ mit } \epsilon = 3$ wahre Aussage

\end{document}
