\documentclass[12pt]{scrartcl}%{article} % Beginn der LaTeX-Datei

%% twocolumn

\usepackage{amsmath,amssymb}  % erleichtert Mathe 
\usepackage{enumerate}% schicke Nummerierung

\usepackage{graphicx} % für Grafik-Einbindung
%\usepackage{hyperref}

\usepackage[ngerman]{babel}
\usepackage[T1]{fontenc}
\usepackage{lmodern}
\usepackage[boxed]{algorithm2e}
\usepackage{parskip}
 % Einstellungen, wenn man deutsch schreiben will, z.B. Trennregeln
\usepackage[utf8]{inputenc}  % für Unix-Systeme
  % ermöglicht die direkte Eingabe von Umlauten und ß
  % evt. obige Zeile ersetzen durch
  % \usepackage[ansinew]{inputenc}  % für Windows
  % \usepackage[applemac]{inputenc} % für den Mac


%%%%%%%%%%%%%%%%%%%%%%%%%%%%%%%%%%%%%%%%%%%%%%%%%%%%%%%%%%%%%%%%%%
%
%  ntheorem
%
\usepackage[thmmarks,amsmath,hyperref,noconfig]{ntheorem} 
  % erlaubt es, Sätze, Definitionen etc. einfach durchzunummerieren.
\newtheorem{satz}{Satz}[section] % Nummerierung nach Abschnitten
\newtheorem{hilfssatz}[satz]{Hilfssatz}
\newtheorem{kor}[satz]{Korollar}

\theorembodyfont{\upshape}
\newtheorem{beispiel}[satz]{Beispiel}
\newtheorem{bemerkung}[satz]{Bemerkung}
\newtheorem{definition}[satz]{Definition} %[section]

\theoremstyle{nonumberplain}
\theoremheaderfont{\itshape}
\theorembodyfont{\normalfont}
\theoremseparator{.}
\theoremsymbol{\ensuremath{_\blacksquare}}
\newtheorem{beweis}{Beweis}
\qedsymbol{\ensuremath{_\blacksquare}}
%\theoremclass{LaTeX}
%
% Ende ntheorem
%
%%%%%%%%%%%%%%%%%%%%%%%%%%%%%%%%%%%%%%%%%%%%%%%%%%%%%%%%%%%%%%%%%%


%\pagestyle{empty}
%
% Ändern der bedruckten Fläche der Seite
% \addtolength{\textwidth}{3cm}  % Befehl mit zwei Argumenten
% \addtolength{\textheight}{3cm}
% \hoffset-2cm % verschiebt das Textfenster nach links
% \voffset-5mm % verschiebt das Textfenster nach oben
%
\parindent=0pt %% keine Einzug zu Beginn von Abs\"atzen
%\parskip=2mm   %% erzeugt einen zusätzliche Zeilenabstand zwischen
                %% Absätzen. Nötig bei \parindent=0pt


%%%%%%%%%%%%%%%%%%%%%%%%%%%%%%%%%%%%%%%%%%%%%%%%%%%%%%%%%%%%%%%%%%
% ermöglicht, farbigen Text zu drucken.
\usepackage{color}
% Und nun werden die Farben definiert - daran können Sie nach Belieben selber rumspielen.
\definecolor{white}{rgb}{1,1,1}
\definecolor{darkred}{rgb}{0.3,0,0}
\definecolor{darkgreen}{rgb}{0,0.3,0}
\definecolor{darkblue}{rgb}{0,0,0.3}
\definecolor{pink}{rgb}{0.78,0.09,0.51}
\definecolor{purple}{rgb}{0.28,0.24,0.55}
\definecolor{orange}{rgb}{1,0.6,0.0}
\definecolor{grey}{rgb}{0.4,0.4,0.4}
\definecolor{aquamarine}{rgb}{0.4,0.8,0.65}


\DeclareMathOperator{\GL}{GL} % einige Macro, siehe am Ende Abschn. 2
\newcommand{\N}{\mathbb{N}}
\newcommand{\Z}{\mathbb{Z}}
\newcommand{\Q}{\mathbb{Q}}
\newcommand{\R}{\mathbb{R}}
\newcommand{\C}{\mathbb{C}}
\newcommand{\cP}{{\mathcal P}}
\newcommand{\bO}[1]{\mathcal O(#1)}
\newcommand{\bT}[1]{\Theta (#1)}

\begin{document}

\author{Dennis Hempfing, Sebastian Koall}
\title{Übung 2}
\date{} 
\pagestyle{myheadings}
\markright{\hfill Dennis Hempfing, Sebastian Koall}

\maketitle % erzeugt den Kopf
 
\section*{Aufgabe 2}

Zu zeigen ist

\begin{align}
	\forall n \ge 6: \frac{n^n}{3^n} \le n! \le \frac{n^n}{2^n}
\end{align}

Induktionsanfang für $n = 6$:

\begin{align*}
	\frac{6^6}{3^6} \le 6! \le \frac{6^6}{2^6} = 64 \le 720 \le 729
\end{align*}

Behauptung: wenn (1) für $n$ gilt, so gilt es auch für $n+1$

\begin{align}
	(\frac{n+1}{3})^{n+1} \le (n+1)! \le (\frac{n+1}{2})^{n+1}
\end{align}

Teile (2) durch (1):

\begin{align}
	\frac{(\frac{n+1}{3})^{n+1}}{(\frac{n}{3})^{n}} \le \frac{(n+1)!}{(n)!} \le \frac{(\frac{n+1}{2})^{n+1}}{(\frac{n}{2})^{n}}
\end{align}

Als nächstes Verringern wir die Basis des Zählers auf $n$, um in Nenner und Zähler den gleichen Exponenten zu haben und können anschließend kürzen.

\begin{align*}
	\frac{(\frac{n+1}{3})^{n}*\frac{n+1}{3}}{(\frac{n}{3})^{n}} \le \frac{(n+1)!}{(n)!} \le \frac{(\frac{n+1}{2})^{n}*\frac{n+1}{2}}{(\frac{n}{2})^{n}}
\end{align*}

\begin{align*}
	(\frac{n+1}{n})^{n} * (\frac{n+1}{3}) \le n+1 \le (\frac{n+1}{n})^{n} * (\frac{n+1}{2})
\end{align*}

Da $n+1$ in allen Teilen der Ungleichung ein Faktor ist, kann es gekürzt werden.

\begin{align*}
	(\frac{n+1}{n})^{n} * (\frac{1}{3}) \le 1 \le (\frac{n+1}{n})^{n} * (\frac{1}{2})
\end{align*}

\begin{align*}
	(1 + \frac{1}{n})^{n} * (\frac{1}{3}) \le 1 \le (1 + \frac{1}{n})^{n} * (\frac{1}{2})
\end{align*}

Der Term $(1 + \frac{1}{n})^{n}$ läuft für $n \rightarrow \infty$ gegen $e$. Also können wir $(1 + \frac{1}{n})^{n}$ durch $e$ ersetzen.

\begin{align*}
	\frac{e}{3} \le 1 \le \frac{e}{2}
\end{align*}

Somit ist die Aussage (1) bewiesen.

$\hfill \square$

\end{document}
