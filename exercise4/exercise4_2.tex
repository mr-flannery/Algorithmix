\documentclass[12pt]{scrartcl}%{article} % Beginn der LaTeX-Datei

%% twocolumn

\usepackage{amsmath,amssymb}  % erleichtert Mathe 
\usepackage{enumerate}% schicke Nummerierung

\usepackage{graphicx} % für Grafik-Einbindung
%\usepackage{hyperref}

\usepackage[ngerman]{babel}
\usepackage[T1]{fontenc}
\usepackage{lmodern}
\usepackage[boxruled]{algorithm2e}
\usepackage{parskip}
 % Einstellungen, wenn man deutsch schreiben will, z.B. Trennregeln
\usepackage[utf8]{inputenc}  % für Unix-Systeme
  % ermöglicht die direkte Eingabe von Umlauten und ß
  % evt. obige Zeile ersetzen durch
  % \usepackage[ansinew]{inputenc}  % für Windows
  % \usepackage[applemac]{inputenc} % für den Mac
\usepackage{listings}

%%%%%%%%%%%%%%%%%%%%%%%%%%%%%%%%%%%%%%%%%%%%%%%%%%%%%%%%%%%%%%%%%%
%
%  ntheorem
%
\usepackage[thmmarks,amsmath,hyperref,noconfig]{ntheorem} 
  % erlaubt es, Sätze, Definitionen etc. einfach durchzunummerieren.
\newtheorem{satz}{Satz}[section] % Nummerierung nach Abschnitten
\newtheorem{hilfssatz}[satz]{Hilfssatz}
\newtheorem{kor}[satz]{Korollar}

\theorembodyfont{\upshape}
\newtheorem{beispiel}[satz]{Beispiel}
\newtheorem{bemerkung}[satz]{Bemerkung}
\newtheorem{definition}[satz]{Definition} %[section]

\theoremstyle{nonumberplain}
\theoremheaderfont{\itshape}
\theorembodyfont{\normalfont}
\theoremseparator{.}
\theoremsymbol{\ensuremath{_\blacksquare}}
\newtheorem{beweis}{Beweis}
\qedsymbol{\ensuremath{_\blacksquare}}
%\theoremclass{LaTeX}
%
% Ende ntheorem
%
%%%%%%%%%%%%%%%%%%%%%%%%%%%%%%%%%%%%%%%%%%%%%%%%%%%%%%%%%%%%%%%%%%


%\pagestyle{empty}
%
% Ändern der bedruckten Fläche der Seite
% \addtolength{\textwidth}{3cm}  % Befehl mit zwei Argumenten
% \addtolength{\textheight}{3cm}
% \hoffset-2cm % verschiebt das Textfenster nach links
% \voffset-5mm % verschiebt das Textfenster nach oben
%
\parindent=0pt %% keine Einzug zu Beginn von Abs\"atzen
%\parskip=2mm   %% erzeugt einen zusätzliche Zeilenabstand zwischen
                %% Absätzen. Nötig bei \parindent=0pt


%%%%%%%%%%%%%%%%%%%%%%%%%%%%%%%%%%%%%%%%%%%%%%%%%%%%%%%%%%%%%%%%%%
% ermöglicht, farbigen Text zu drucken.
\usepackage{color}
% Und nun werden die Farben definiert - daran können Sie nach Belieben selber rumspielen.
\definecolor{white}{rgb}{1,1,1}
\definecolor{darkred}{rgb}{0.3,0,0}
\definecolor{darkgreen}{rgb}{0,0.3,0}
\definecolor{darkblue}{rgb}{0,0,0.3}
\definecolor{pink}{rgb}{0.78,0.09,0.51}
\definecolor{purple}{rgb}{0.28,0.24,0.55}
\definecolor{orange}{rgb}{1,0.6,0.0}
\definecolor{grey}{rgb}{0.4,0.4,0.4}
\definecolor{aquamarine}{rgb}{0.4,0.8,0.65}


\DeclareMathOperator{\GL}{GL} % einige Macro, siehe am Ende Abschn. 2
\newcommand{\N}{\mathbb{N}}
\newcommand{\Z}{\mathbb{Z}}
\newcommand{\Q}{\mathbb{Q}}
\newcommand{\R}{\mathbb{R}}
\newcommand{\C}{\mathbb{C}}
\newcommand{\cP}{{\mathcal P}}
\newcommand{\bO}[1]{\mathcal O(#1)}
\newcommand{\bT}[1]{\Theta (#1)}
\newcommand{\bOg}[1]{\Omega (#1)}

% Code-Macro, wenn man mal etwas als Code highlighten möchte
\newcommand{\code}[1]{\lstinline[basicstyle=\ttfamily\color{black}]{#1}}

\begin{document}

\author{Dennis Hempfing, Sebastian Koall}
\title{Übung 4}
\date{} 
\pagestyle{myheadings}
\markright{\hfill Dennis Hempfing, Sebastian Koall}

\maketitle % erzeugt den Kopf

\section*{Aufgabe 2}

Wir nutzen jeweils einen Stack um die Operationen \code{enqueue} bzw. \code{dequeue} zu implementieren. Das heißt wir nennen einen Stack den \code{enqueue-Stack} und den anderen den \code{dequeue-Stack}.

\code{enqueue} und \code{dequeue} können grundsätzlich in konstanter Zeit ausgeführt werden. Bei jeder \code{enqueue}-Operationen wird ein Element auf den \code{enqueue-Stack} gelegt und bei jeder \code{dequeue}-Operation wird ein Element vom \code{dequeue-Stack} entfernt. Sollte bei Aufruf der \code{dequeue}-Operation der \code{dequeue-Stack} leer sein, müssen zunächst alle Elemente des \code{enqueue-Stacks} auf den \code{dequeue-Stack} gelegt werden (im Folgenden \code{shift} genannt). Jedes Element muss also vom \code{enqueue-Stack} mittels \code{pop} entfernt und mittels \code{push} auf den \code{dequeue-Stack} gelegt werden. Wir gehen dabei davon aus, dass \code{pop} und \code{push} ebenfalls in konstanter Zeit ausgeführt werden können. Liegen $n$ Elemente auf dem \code{enqueue-Stack}, kostet die \code{shift}-Operation also genau $2*n$.

Wir betrachtetn nun die amortisierte Laufzeit dieser Queue-Implementierung mithilfe der Guthabenmethode. Dazu wählen wir folgende Besteuerungsmethode: Um die Kosten der \code{shift}-Operation zu decken, zahlt \code{enqueue} bei jedem Aufruf $2c$ auf das Guthabenkonto ein. Daraus ergibt sich folgende Kostengleichung: 

\begin{align*}
	a_i = 3*e_i + d_i + 0*s_i
\end{align*}

Dabei sind $e_i$ die Kosten der \code{enqueue}-Operation, $d_i$ die Kosten der \code{dequeue}-Operation und $s_i$ die Kosten der \code{shift}-Operation.

Nun bleibt zu zeigen, dass $a_0 = 0$ (1) und $a_i \ge 0$ (2) gelten. (1) gilt offensichtlich, da bei einer Folge von 0 Operationen kein Guthaben eingezahlt wird. (2) gilt ebenfalls, da \code{shift} genau zwei mal soviel Guthaben verbraucht, wie \code{enqueue}-Operationen seit dem letzten \code{shift} ausgeführt worden sind. Da jedes \code{enqueue} genau $2c$ einzahlt, ist das Guthaben nach jedem \code{shift} genau 0.

Für eine belieibige Folge von $n$ Operationen ergibt sich somit eine Gesamtlaufzeit von

\begin{align*}
	T(n) = 3*n + n + 0*n = \bO{4n} = \bO{n}
\end{align*}
$\hfill \square$

\end{document}
