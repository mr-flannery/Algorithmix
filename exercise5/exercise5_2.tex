\documentclass[12pt]{scrartcl}%{article} % Beginn der LaTeX-Datei

%% twocolumn

\usepackage{amsmath,amssymb}  % erleichtert Mathe 
\usepackage{enumerate}% schicke Nummerierung

\usepackage{graphicx} % für Grafik-Einbindung
%\usepackage{hyperref}

\usepackage[ngerman]{babel}
\usepackage[T1]{fontenc}
\usepackage{lmodern}
\usepackage[boxruled]{algorithm2e}
\usepackage{parskip}
 % Einstellungen, wenn man deutsch schreiben will, z.B. Trennregeln
\usepackage[utf8]{inputenc}  % für Unix-Systeme
  % ermöglicht die direkte Eingabe von Umlauten und ß
  % evt. obige Zeile ersetzen durch
  % \usepackage[ansinew]{inputenc}  % für Windows
  % \usepackage[applemac]{inputenc} % für den Mac
\usepackage{listings}

%%%%%%%%%%%%%%%%%%%%%%%%%%%%%%%%%%%%%%%%%%%%%%%%%%%%%%%%%%%%%%%%%%
%
%  ntheorem
%
\usepackage[thmmarks,amsmath,hyperref,noconfig]{ntheorem} 
  % erlaubt es, Sätze, Definitionen etc. einfach durchzunummerieren.
\newtheorem{satz}{Satz}[section] % Nummerierung nach Abschnitten
\newtheorem{hilfssatz}[satz]{Hilfssatz}
\newtheorem{kor}[satz]{Korollar}

\theorembodyfont{\upshape}
\newtheorem{beispiel}[satz]{Beispiel}
\newtheorem{bemerkung}[satz]{Bemerkung}
\newtheorem{definition}[satz]{Definition} %[section]

\theoremstyle{nonumberplain}
\theoremheaderfont{\itshape}
\theorembodyfont{\normalfont}
\theoremseparator{.}
\theoremsymbol{\ensuremath{_\blacksquare}}
\newtheorem{beweis}{Beweis}
\qedsymbol{\ensuremath{_\blacksquare}}
%\theoremclass{LaTeX}
%
% Ende ntheorem
%
%%%%%%%%%%%%%%%%%%%%%%%%%%%%%%%%%%%%%%%%%%%%%%%%%%%%%%%%%%%%%%%%%%


%\pagestyle{empty}
%
% Ändern der bedruckten Fläche der Seite
% \addtolength{\textwidth}{3cm}  % Befehl mit zwei Argumenten
% \addtolength{\textheight}{3cm}
% \hoffset-2cm % verschiebt das Textfenster nach links
% \voffset-5mm % verschiebt das Textfenster nach oben
%
\parindent=0pt %% keine Einzug zu Beginn von Abs\"atzen
\parskip=0.5mm   %% erzeugt einen zusätzliche Zeilenabstand zwischen
                %% Absätzen. Nötig bei \parindent=0pt


%%%%%%%%%%%%%%%%%%%%%%%%%%%%%%%%%%%%%%%%%%%%%%%%%%%%%%%%%%%%%%%%%%
% ermöglicht, farbigen Text zu drucken.
\usepackage{color}
% Und nun werden die Farben definiert - daran können Sie nach Belieben selber rumspielen.
\definecolor{white}{rgb}{1,1,1}
\definecolor{darkred}{rgb}{0.3,0,0}
\definecolor{darkgreen}{rgb}{0,0.3,0}
\definecolor{darkblue}{rgb}{0,0,0.3}
\definecolor{pink}{rgb}{0.78,0.09,0.51}
\definecolor{purple}{rgb}{0.28,0.24,0.55}
\definecolor{orange}{rgb}{1,0.6,0.0}
\definecolor{grey}{rgb}{0.4,0.4,0.4}
\definecolor{aquamarine}{rgb}{0.4,0.8,0.65}


\DeclareMathOperator{\GL}{GL} % einige Macro, siehe am Ende Abschn. 2
\newcommand{\N}{\mathbb{N}}
\newcommand{\Z}{\mathbb{Z}}
\newcommand{\Q}{\mathbb{Q}}
\newcommand{\R}{\mathbb{R}}
\newcommand{\C}{\mathbb{C}}
\newcommand{\cP}{{\mathcal P}}
\newcommand{\bO}[1]{\mathcal O(#1)}
\newcommand{\bT}[1]{\Theta (#1)}
\newcommand{\bOg}[1]{\Omega (#1)}

% Code-Macro, wenn man mal etwas als Code highlighten möchte
\newcommand{\code}[1]{\lstinline[basicstyle=\ttfamily\color{black}]{#1}}

\begin{document}

\author{Dennis Hempfing, Sebastian Koall}
\title{Übung 5}
\date{} 
\pagestyle{myheadings}
\markright{\hfill Dennis Hempfing, Sebastian Koall}

\maketitle % erzeugt den Kopf

\section*{Aufgabe 2}

Um zu zeigen, dass der Algorithmus korrekt ist, stellen wir folgende Korrektheitsbedingung auf:

\begin{itemize}
	\item[1)] Der Algorithmus terminiert.
	\item[2)] Wenn $\exists i \le n \forall j \le n: M[i,j] = 1 \wedge i \not= j$ gilt, soll der Algorithmus $yes$ zurückgeben.\\
\end{itemize}

Dass der Algorithmus terminert ist leich erkennbar. Beide Schleifen iterieren von 2 bzw. von 1 bis $n$, es werden also maximal $2n-3$ Schleifendurchläufe durchgeführt. Alle anderen Schritte werden genau einmal ausgeführt. 

Die zweite Bedingung wird von der zweiten Schleife des Algorithmus geprüft. Um zu zeigen, dass die zweite Schleife das korrekte Ergebnis liefert, muss gezeigt werden, dass die erste Schleife ein Element findet, dass die SuperWinner-Bedingung erfüllen kann, also eine Mannschaft, die alle betrachteten Spiele gewonnen hat.

Dazu nutzen wir folgende Schleifeninvariante:

\begin{align}
	\forall k \in V: k < j \wedge k \not= i: k \not= SuperWinner
\end{align}

Zuerst wird gezeigt, dass (1) zu Beginn der Schleife gilt. Inital gelten $i=1$ und $j=2$. Somit gibt es kein $k \in V$, dass die Bedingung erfüllt. (1) gilt also.

Anschließend soll die Gültigkeit von (1) während der Ausführung der Schleife gezeigt werden. Dazu werden folgende Fälle betrachtet:

\paragraph{Fall 1:} $k = i$

Für diesen Fall ist die Voraussetzung für (1) nicht erfüllt, (1) ist also gültig.

\paragraph{Fall 2:} $k < i$

\subparagraph{Fall 2.1:}

Der Algorithmus hat Zeile $k$ bereits besucht. Da $k \not= i$ muss also $\exists y \in V: A[k,y] = 0$ gelten. $k$ hat also bereits ein Spiel verloren und kann die SuperWinner-Bedingung nicht erfüllen.

\subparagraph{Fall 2.2:}

Der Algorithmus hat Zeile $k$ nicht besucht. Zeile $k$ wurde also übersprungen da $\exists y \in V: A[y,k] = 1$ gilt. $k$ hat also bereits ein Spiel verloren und kann die SuperWinner-Bedingung nicht erfüllen.

\paragraph{Fall 3:} $k > i$

Zu Beginn jeder Schleife gilt $j > i$. Zusammen mit $k > i$ müsste also $j > k > i$ gelten. Dies heißt, dass $k$ bereits ein Spiel verloren haben muss, da sonst $k = i$ gelten müsste. Also gilt (1) auch für diesen Fall.

Nach Abschluss der Schleife gilt $k = j$, die Bedingung ist nicht erfüllt und (1) gilt.

Somit ist gezeigt, dass die erste Schleife ein korrektes Ergebnis für die zweite Schleife liefert, die somit die zweite Korrektheitsbedingung erfüllt. 

$\hfill\square$

\end{document}
