\documentclass[12pt]{scrartcl}%{article} % Beginn der LaTeX-Datei

%% twocolumn

\usepackage{amsmath,amssymb}  % erleichtert Mathe 
\usepackage{enumerate}% schicke Nummerierung

\usepackage{graphicx} % für Grafik-Einbindung
%\usepackage{hyperref}

\usepackage[ngerman]{babel}
\usepackage[T1]{fontenc}
\usepackage{lmodern}
\usepackage[boxed]{algorithm2e}
\usepackage{parskip}
\usepackage{listings}
 % Einstellungen, wenn man deutsch schreiben will, z.B. Trennregeln
\usepackage[utf8]{inputenc}  % für Unix-Systeme
  % ermöglicht die direkte Eingabe von Umlauten und ß
  % evt. obige Zeile ersetzen durch
  % \usepackage[ansinew]{inputenc}  % für Windows
  % \usepackage[applemac]{inputenc} % für den Mac


%%%%%%%%%%%%%%%%%%%%%%%%%%%%%%%%%%%%%%%%%%%%%%%%%%%%%%%%%%%%%%%%%%
%
%  ntheorem
%
\usepackage[thmmarks,amsmath,hyperref,noconfig]{ntheorem} 
  % erlaubt es, Sätze, Definitionen etc. einfach durchzunummerieren.
\newtheorem{satz}{Satz}[section] % Nummerierung nach Abschnitten
\newtheorem{hilfssatz}[satz]{Hilfssatz}
\newtheorem{kor}[satz]{Korollar}

\theorembodyfont{\upshape}
\newtheorem{beispiel}[satz]{Beispiel}
\newtheorem{bemerkung}[satz]{Bemerkung}
\newtheorem{definition}[satz]{Definition} %[section]

\theoremstyle{nonumberplain}
\theoremheaderfont{\itshape}
\theorembodyfont{\normalfont}
\theoremseparator{.}
\theoremsymbol{\ensuremath{_\blacksquare}}
\newtheorem{beweis}{Beweis}
\qedsymbol{\ensuremath{_\blacksquare}}
%\theoremclass{LaTeX}
%
% Ende ntheorem
%
%%%%%%%%%%%%%%%%%%%%%%%%%%%%%%%%%%%%%%%%%%%%%%%%%%%%%%%%%%%%%%%%%%


%\pagestyle{empty}
%
% Ändern der bedruckten Fläche der Seite
% \addtolength{\textwidth}{3cm}  % Befehl mit zwei Argumenten
% \addtolength{\textheight}{3cm}
% \hoffset-2cm % verschiebt das Textfenster nach links
% \voffset-5mm % verschiebt das Textfenster nach oben
%
\parindent=0pt %% keine Einzug zu Beginn von Abs\"atzen
%\parskip=2mm   %% erzeugt einen zusätzliche Zeilenabstand zwischen
                %% Absätzen. Nötig bei \parindent=0pt


%%%%%%%%%%%%%%%%%%%%%%%%%%%%%%%%%%%%%%%%%%%%%%%%%%%%%%%%%%%%%%%%%%
% ermöglicht, farbigen Text zu drucken.
\usepackage{color}
% Und nun werden die Farben definiert - daran können Sie nach Belieben selber rumspielen.
\definecolor{white}{rgb}{1,1,1}
\definecolor{darkred}{rgb}{0.3,0,0}
\definecolor{darkgreen}{rgb}{0,0.3,0}
\definecolor{darkblue}{rgb}{0,0,0.3}
\definecolor{pink}{rgb}{0.78,0.09,0.51}
\definecolor{purple}{rgb}{0.28,0.24,0.55}
\definecolor{orange}{rgb}{1,0.6,0.0}
\definecolor{grey}{rgb}{0.4,0.4,0.4}
\definecolor{aquamarine}{rgb}{0.4,0.8,0.65}


\DeclareMathOperator{\GL}{GL} % einige Macro, siehe am Ende Abschn. 2
\newcommand{\N}{\mathbb{N}}
\newcommand{\Z}{\mathbb{Z}}
\newcommand{\Q}{\mathbb{Q}}
\newcommand{\R}{\mathbb{R}}
\newcommand{\C}{\mathbb{C}}
\newcommand{\cP}{{\mathcal P}}
\newcommand{\bO}[1]{\mathcal O(#1)}
\newcommand{\bT}[1]{\Theta (#1)}
\newcommand{\bOg}[1]{\Omega (#1)}

\newcommand{\code}[1]{\lstinline[basicstyle=\ttfamily\color{black}]{#1}}

\begin{document}

\author{Dennis Hempfing, Sebastian Koall}
\title{Übung 5}
\date{} 
\pagestyle{myheadings}
\markright{\hfill Dennis Hempfing, Sebastian Koall}

\maketitle % erzeugt den Kopf
 
\section*{Aufgabe 4}
\subsubsection*{Idee}
Grundsätzlich funktioniert der Algorithmus wie Mergesort, mit den Ausnahmen, dass wir einen zusätzlichen Array statt der Teilarrays erstellen und uns Positionen im gesamten Array merken um die einzelnen Teile zu mergen.

Die Funktion \code{unsortedness} ist in zwei Varianten vorhanden, um mit der ersten das Interface für den Aufruf bereitzustellen und einen Hilfsarray anzulegen. Die eigentliche \code{\_unsortedness} Funktion entspricht dem Mergesort Algorithmus mit der Ausnahme, dass keine sortierten Arrays sondern die Unsortiertheit des Arrays des betrachteten Teils des Arrays zurückgegeben wird. Die erste Schleife in der Funktion \code{Merge} sichert den aktuellen Zustand des Arrays. In der zweiten Schleife wird über den aktuellen Teilbereich des Arrays iteriert, um die in ihm enthaltenen Teilarrays zu mergen. Die ersten beiden Fälle der Fallunterscheidung fangen die Fälle ab, dass entweder der linke oder rechte Teilarray bereits leer sind und sich im jeweils anderen noch ein Element befindet (Mehr ist nicht möglich, da wir immer in der Mitte teilen). Die letzten beiden Fälle führen den Hauptbestandteil des Mergens aus. Interessant ist der Fall, dass ein Element in dem rechten Teilarray gemergt wird. In diesem Fall muss das Element an X größeren Elementen vorbeigemergt werden. X ist dabei die Anzahl der noch nicht gemergten Elemente im linken Teilarray (mid - i + 1). Nach der Rückgabe werden alle Unsortedness Summanden summiert um die Unsortedness des gesamten Arrays zu erhalten (Nebenbei liegt der Array a nun sortiert vor).

\subsubsection*{Algorithmus}

\begin{algorithm}[H]
	\caption{unsortedness}
	\SetKwInOut{Input}{Eingabe}\SetKwInOut{Output}{Ausgabe}
	
	\Input{Array a}
	\Output{Unsortiertheit n des Array a}
	\BlankLine
	a' = Array[a.length()]\;
	mid = min + floor((max - min) / 2)\;
	unsortedness = \_unsortedness(a, a', 1, a.length(), mid)\;	
	\BlankLine
	return unsortedness\;	
\end{algorithm}

\begin{algorithm}[H]
	\caption{\_unsortedness}
	\SetKwInOut{Input}{Eingabe}\SetKwInOut{Output}{Ausgabe}
	
	\Input{Array a, Array a', min, max}
	\Output{Unsortiertheit n von min bis max}
	\BlankLine
	\If {(min = max)} { return 0\; }
	
	mid = min + floor((max - min) / 2)\;
	unsortedness = \_unsortedness(a, a', min, mid)\;
	unsortedness += \_unsortedness(a, a', mid + 1, max)\;
	unsortedness += merge(a, a', min, max)\;
	\BlankLine
	return unsortedness\;
\end{algorithm}

\begin{algorithm}[H]
	\caption{merge}
	\SetKwInOut{Input}{Eingabe}\SetKwInOut{Output}{Ausgabe}
	
	\Input{Array a, Array a', min, max, mid}
	\Output{Unsortiertheit n von min bis max}
	\BlankLine
	\For {(k = min to max)} {\
		a'[k] = a[k]\;
	}
	unsortedness = 0\;
	\For {(k = min to max)} {\
		\If {(i > mid)} {\
			a[k] = a'[j]\;
			j += 1;
		}
		\ElseIf {(j > max)} {\
			a[k] = a'[i]\;
			i += 1\;
		}
		\ElseIf {(a'[j] < a'[i])} {\
			a[k] = a'[j]\;
			j += 1\;
			unsortedness += (mid - i + 1)\;
		}
		\Else {\
			a[k] = a'[i]\;
			i += 1\;
		}
	}
	\BlankLine
	return unsortedness\;
\end{algorithm}

\subsubsection*{Korrektheit}
Korrektheitsbedingungen:
\newline
(1) unsortedness terminiert
\newline
(2) unsortedness gibt die Unsortiertheit von des Eingabearrays zurück

(1) ist erfüllt, weil nach log(n) Schritten zuerst alle Teilbereiche des Arrays nur noch die Länge 1 besitzen und anschließend über jeden der erzeugten Teilbereiche zwei Mal iteriert wird.

Für (2) muss gezeigt werden, dass eine Schleifeninvariante für den angepassten Mergesort existiert und dass die zurückgegebene Unsortiertheit des Teilbereichs korrekt ist. Ist die Unsortiertheit für einen gemergten Teilbereich korrekt, so stimmt sie auch in jedem weiteren Mergeschritt. Die Schleifeninvariante lautet wie folgt:

$ A \text{ (Vor der Schleife) } \forall i: i = k, a[i] \le a[k]$ \\
$ B \text{ (Schleifeninvariante) } \forall i: min \le i \le k, a[min] \le a[i] \le a[k]$ \\
$ B \text{ (Schleifenende) } \forall i: min \le i \le max, a[min] \le a[i] \le a[max]$

Am Schleifenende wurden alle Elemente in a überschrieben, da am Anfang eine Kopie von a gemacht wird und anschließend von der Kopie Elemente in der Art und Weise zurückkopiert werden, dass ($ min \le i \le mid < j \le max $) jedes Element lediglich einmal an eine andere Stelle geschrieben wurde. Es handelt sich deshalb bei a nach dem Merge nur um eine Permutation seiner selbst. Die Unsortiertheit wird genau dann um X Elemente erhöht, wenn sie vom rechten Teil merged während der linke nicht leer ist. X entspricht der Anzahl der Elemente im linken Teil, welche noch nicht gemerged wurden ($ mid - 1 + 1 $). In den Teilbereichen liegen zwei sortierte Teile vor, deshalb muss, wenn ein Element von rechts vor X Elementen von links gemergt wird, die Unsortiertheit des Elementes bezüglich des betrachteten Teilbereichs X betragen.

Da der Algorithmus terminiert, eine Schleifeninvariante für den angepassten Mergesort existiert und aufbauend auf diesem Mergesort eine Zahlweise der Unsortheit gezeigt wurde, ist somit die Korrektheit des Algorithmus gezeigt.

$\hfill \square$

\subsubsection*{Laufzeit}
Äquivalent zur Laufzeit des Mergesort gilt für den Aufwand die Teilarrays zu sortieren $ T(\lfloor \frac{n}{2} \rfloor) + T(\lceil \frac{n}{2} \rceil) $. In der Mergephase jedoch wird zweimal über die komplette Länge des Arrays iteriert, was folgenden Laufzeit $ T(n) = T(\lfloor \frac{n}{2} \rfloor) + 2 * n $ ergibt. Nach dem 2. Fall ergibt sich folgende Worst-Case Laufzeit: $ \bO{2*n*\log {n}} = \bO{n*\log {n}} $.

\end{document}
