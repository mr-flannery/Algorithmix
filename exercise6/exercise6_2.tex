\documentclass[12pt]{scrartcl}%{article} % Beginn der LaTeX-Datei

%% twocolumn

\usepackage{amsmath,amssymb}  % erleichtert Mathe 
\usepackage{enumerate}% schicke Nummerierung

\usepackage{graphicx} % für Grafik-Einbindung
%\usepackage{hyperref}

\usepackage[ngerman]{babel}
\usepackage[T1]{fontenc}
\usepackage{lmodern}
\usepackage[boxruled]{algorithm2e}
\usepackage{parskip}
 % Einstellungen, wenn man deutsch schreiben will, z.B. Trennregeln
\usepackage[utf8]{inputenc}  % für Unix-Systeme
  % ermöglicht die direkte Eingabe von Umlauten und ß
  % evt. obige Zeile ersetzen durch
  % \usepackage[ansinew]{inputenc}  % für Windows
  % \usepackage[applemac]{inputenc} % für den Mac
\usepackage{listings}

%%%%%%%%%%%%%%%%%%%%%%%%%%%%%%%%%%%%%%%%%%%%%%%%%%%%%%%%%%%%%%%%%%
%
%  ntheorem
%
\usepackage[thmmarks,amsmath,hyperref,noconfig]{ntheorem} 
  % erlaubt es, Sätze, Definitionen etc. einfach durchzunummerieren.
\newtheorem{satz}{Satz}[section] % Nummerierung nach Abschnitten
\newtheorem{hilfssatz}[satz]{Hilfssatz}
\newtheorem{kor}[satz]{Korollar}

\theorembodyfont{\upshape}
\newtheorem{beispiel}[satz]{Beispiel}
\newtheorem{bemerkung}[satz]{Bemerkung}
\newtheorem{definition}[satz]{Definition} %[section]

\theoremstyle{nonumberplain}
\theoremheaderfont{\itshape}
\theorembodyfont{\normalfont}
\theoremseparator{.}
\theoremsymbol{\ensuremath{_\blacksquare}}
\newtheorem{beweis}{Beweis}
\qedsymbol{\ensuremath{_\blacksquare}}
%\theoremclass{LaTeX}
%
% Ende ntheorem
%
%%%%%%%%%%%%%%%%%%%%%%%%%%%%%%%%%%%%%%%%%%%%%%%%%%%%%%%%%%%%%%%%%%


%\pagestyle{empty}
%
% Ändern der bedruckten Fläche der Seite
% \addtolength{\textwidth}{3cm}  % Befehl mit zwei Argumenten
% \addtolength{\textheight}{3cm}
% \hoffset-2cm % verschiebt das Textfenster nach links
% \voffset-5mm % verschiebt das Textfenster nach oben
%
\parindent=0pt %% keine Einzug zu Beginn von Abs\"atzen
\parskip=0.5mm   %% erzeugt einen zusätzliche Zeilenabstand zwischen
                %% Absätzen. Nötig bei \parindent=0pt


%%%%%%%%%%%%%%%%%%%%%%%%%%%%%%%%%%%%%%%%%%%%%%%%%%%%%%%%%%%%%%%%%%
% ermöglicht, farbigen Text zu drucken.
\usepackage{color}
% Und nun werden die Farben definiert - daran können Sie nach Belieben selber rumspielen.
\definecolor{white}{rgb}{1,1,1}
\definecolor{darkred}{rgb}{0.3,0,0}
\definecolor{darkgreen}{rgb}{0,0.3,0}
\definecolor{darkblue}{rgb}{0,0,0.3}
\definecolor{pink}{rgb}{0.78,0.09,0.51}
\definecolor{purple}{rgb}{0.28,0.24,0.55}
\definecolor{orange}{rgb}{1,0.6,0.0}
\definecolor{grey}{rgb}{0.4,0.4,0.4}
\definecolor{aquamarine}{rgb}{0.4,0.8,0.65}


\DeclareMathOperator{\GL}{GL} % einige Macro, siehe am Ende Abschn. 2
\newcommand{\N}{\mathbb{N}}
\newcommand{\Z}{\mathbb{Z}}
\newcommand{\Q}{\mathbb{Q}}
\newcommand{\R}{\mathbb{R}}
\newcommand{\C}{\mathbb{C}}
\newcommand{\cP}{{\mathcal P}}
\newcommand{\bO}[1]{\mathcal O(#1)}
\newcommand{\bT}[1]{\Theta (#1)}
\newcommand{\bOg}[1]{\Omega (#1)}

% Code-Macro, wenn man mal etwas als Code highlighten möchte
\newcommand{\code}[1]{\lstinline[basicstyle=\ttfamily\color{black}]{#1}}

\begin{document}

\author{Dennis Hempfing, Sebastian Koall}
\title{Übung 6}
\date{} 
\pagestyle{myheadings}
\markright{\hfill Dennis Hempfing, Sebastian Koall}

\maketitle % erzeugt den Kopf

\section*{Aufgabe 2}

Betrachten wir das Problem zunächst für einen konstanten Index $i$ und einen fortlaufenden Index $j$. Wir wollen also wissen, ob es einen Politiker gibt, der die gleiche Meinung zu Bildungspolitik hat wie $i$. Da wir einen Gegenspieler haben, haben alle Anfragen die Form \glqq Zeige mir für Politiker $i$ einen beliebigen anderen, noch nicht befragten Politiker $j$!\grqq. Nun kann man zwei Fälle unterscheiden. 

Im ersten Fall gibt es keinen Politiker, der die gleiche Meinung wie Politiker $i$. In diesem Fall ist es vom Adversary-Standpunkt aus egal, welchen Politiker er uns zuerst zeigt, da wir für keinen einen Match bekommen. Wir müssen alle $n-1$ Politiker $j$ mit $j \not= i$ befragen, da wir ansonsten einen Match verpassen könnten.

Im zweiten Fall gibt es mindestens einen Match. In diesem Fall wird der Gegenspieler uns erst alle Politiker liefern, die eine andere Meinung als $i$ haben, und uns erst dann einen Match zeigen, wenn es nur noch Matches gibt. Da der Algorithmus nur fragt, ob es mindestens einen Match gibt, besteht der Worst Case aus genau einem Match, da in diesem Fall der Gegenspieler maximal viele Missmatches liefern kann, bevor er einen Match gibt. Also werden auch in diesem Fall $n-1 = \bO{n}$ Anfragen gebraucht.\\

Betrachten wir nun das gleiche Problem für alle Paare von Politikern. Alle Anfragen haben nun die Form \glqq Zeige mir für ein beliebiges, nicht befragtes Paar Politiker $i, j$, ob sie die gleiche Meinung haben!\grqq. Wie zuvor wird der Gegenspieler uns erst alle Missmatches liefern. Wie oben bereits festgestellt ist der Worst Case also, dass es genau einen oder keinen Match gibt. Dann kann der Gegenspieler uns zunächst $(\frac{n(n-1)}{2})-1$ Missmatches liefern, bevor das letzte Paar befragt wird, was entweder ein Match ist oder nicht. Demzufolge werden $\frac{n(n-1)}{2}$ Anfragen benötigt, der Algorithmus hat als eine Laufzeit von $\bO{n^2}$.

$\hfill \square$

\paragraph{Umsetzung als Algorithmus}

Die Idee ist denkbar einfach. Wir iterieren spaltenweise über die Matrix und fragen für jeden Politiker in der jeweiligen Spalte, wenn er mit dem aktuell betrachteten Politker ein unbefragtes Paar ergibt (was über die Wahl der Zählvariablen sichergestellt werden kann), ob sie die gleiche Meinung zu Bildungspolitik haben ($A[i,j] = 1$). Sobald dies einmal eintritt, wird $true$ returnend, sonst $false$.

\begin{algorithm}[!ht]
	\LinesNumbered
	\caption{repraesentationDerBildungspolitikdebatte}
	\SetKwInOut{Input}{Eingabe}\SetKwInOut{Output}{Ausgabe}
	
	\Input{Matrix $A$ der Größe $n^2$ für $n$ Politiker}
	\Output{Gibt es zwei Politiker mit der gleichen Meinung zur Bildungspolitik}
	\BlankLine
	\For{i = 1 to n} {
		\For{j = 2 + i to n} {
			\If{A(i,j) == 1} {
				return true\;		
			}	
		}
	}
	return false\;
\end{algorithm}

\paragraph{Korrektheit}

Der Algorithmus ist korrekt, wenn er terminiert und für einen Eintrag $A(i,j) = 1$ true zrürckgibt. Dass er terminiert ist offensichtlich, da lediglich in den beiden for-Schleifen von 1 bis n iteriert wird. In Zeile 3 wird genau die zweite Korrektheisbedingung geprüft. Trifft sie ein, wird in Zeile 4 true zurückgegeben. Der Algorithmus ist also korrekt.

\paragraph{Laufzeit}

Die Laufzeit wird durch die beiden verschachtelten for-Schleifen gegeben, da alle anderen Aufrufe in konstanter Zeit erfolgen. Durch die Konstruktion der for-Schleifen werden geanu $\frac{n(n-1)}{2}$ Schritte ausgeführt, die Laufzeit des Algorithmus beträgt also $\bO{n^2}$.

\end{document}
