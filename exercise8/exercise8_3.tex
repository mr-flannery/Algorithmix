\documentclass[12pt]{scrartcl}%{article} % Beginn der LaTeX-Datei

%% twocolumn

\usepackage{amsmath,amssymb}  % erleichtert Mathe 
\usepackage{enumerate}% schicke Nummerierung

\usepackage{graphicx} % für Grafik-Einbindung
%\usepackage{hyperref}

\usepackage[ngerman]{babel}
\usepackage[T1]{fontenc}
\usepackage{lmodern}
\usepackage[boxruled]{algorithm2e}
\usepackage{parskip}
 % Einstellungen, wenn man deutsch schreiben will, z.B. Trennregeln
\usepackage[utf8]{inputenc}  % für Unix-Systeme
  % ermöglicht die direkte Eingabe von Umlauten und ß
  % evt. obige Zeile ersetzen durch
  % \usepackage[ansinew]{inputenc}  % für Windows
  % \usepackage[applemac]{inputenc} % für den Mac
\usepackage{listings}

%%%%%%%%%%%%%%%%%%%%%%%%%%%%%%%%%%%%%%%%%%%%%%%%%%%%%%%%%%%%%%%%%%
%
%  ntheorem
%
\usepackage[thmmarks,amsmath,hyperref,noconfig]{ntheorem} 
  % erlaubt es, Sätze, Definitionen etc. einfach durchzunummerieren.
\newtheorem{satz}{Satz}[section] % Nummerierung nach Abschnitten
\newtheorem{hilfssatz}[satz]{Hilfssatz}
\newtheorem{kor}[satz]{Korollar}

\theorembodyfont{\upshape}
\newtheorem{beispiel}[satz]{Beispiel}
\newtheorem{bemerkung}[satz]{Bemerkung}
\newtheorem{definition}[satz]{Definition} %[section]

\theoremstyle{nonumberplain}
\theoremheaderfont{\itshape}
\theorembodyfont{\normalfont}
\theoremseparator{.}
\theoremsymbol{\ensuremath{_\blacksquare}}
\newtheorem{beweis}{Beweis}
\qedsymbol{\ensuremath{_\blacksquare}}
%\theoremclass{LaTeX}
%
% Ende ntheorem
%
%%%%%%%%%%%%%%%%%%%%%%%%%%%%%%%%%%%%%%%%%%%%%%%%%%%%%%%%%%%%%%%%%%


%\pagestyle{empty}
%
% Ändern der bedruckten Fläche der Seite
% \addtolength{\textwidth}{3cm}  % Befehl mit zwei Argumenten
% \addtolength{\textheight}{3cm}
% \hoffset-2cm % verschiebt das Textfenster nach links
% \voffset-5mm % verschiebt das Textfenster nach oben
%
\parindent=0pt %% keine Einzug zu Beginn von Abs\"atzen
\parskip=0.5mm   %% erzeugt einen zusätzliche Zeilenabstand zwischen
                %% Absätzen. Nötig bei \parindent=0pt


%%%%%%%%%%%%%%%%%%%%%%%%%%%%%%%%%%%%%%%%%%%%%%%%%%%%%%%%%%%%%%%%%%
% ermöglicht, farbigen Text zu drucken.
\usepackage{color}
% Und nun werden die Farben definiert - daran können Sie nach Belieben selber rumspielen.
\definecolor{white}{rgb}{1,1,1}
\definecolor{darkred}{rgb}{0.3,0,0}
\definecolor{darkgreen}{rgb}{0,0.3,0}
\definecolor{darkblue}{rgb}{0,0,0.3}
\definecolor{pink}{rgb}{0.78,0.09,0.51}
\definecolor{purple}{rgb}{0.28,0.24,0.55}
\definecolor{orange}{rgb}{1,0.6,0.0}
\definecolor{grey}{rgb}{0.4,0.4,0.4}
\definecolor{aquamarine}{rgb}{0.4,0.8,0.65}


\DeclareMathOperator{\GL}{GL} % einige Macro, siehe am Ende Abschn. 2
\newcommand{\N}{\mathbb{N}}
\newcommand{\Z}{\mathbb{Z}}
\newcommand{\Q}{\mathbb{Q}}
\newcommand{\R}{\mathbb{R}}
\newcommand{\C}{\mathbb{C}}
\newcommand{\cP}{{\mathcal P}}
\newcommand{\bO}[1]{\mathcal O(#1)}
\newcommand{\bT}[1]{\Theta (#1)}
\newcommand{\bOg}[1]{\Omega (#1)}

% Code-Macro, wenn man mal etwas als Code highlighten möchte
\newcommand{\code}[1]{\lstinline[basicstyle=\ttfamily\color{black}]{#1}}

\begin{document}

\author{Dennis Hempfing, Sebastian Koall}
\title{Übung 8}
\date{} 
\pagestyle{myheadings}
\markright{\hfill Dennis Hempfing, Sebastian Koall}

\maketitle % erzeugt den Kopf

\section*{Aufgabe 3}

Wir wollen zeigen, dass {\scshape SetCover} NP-vollständig ist. Dazu muss zum einen gezeigt werden, dass {\scshape SetCover} in NP liegt und zum anderen, dass {\scshape SetCover} NP-schwer ist.\\

Um zu zeigen, dass {\scshape SetCover} in NP liegt konstruieren wir einen Polynomialzeitalgorithmus, der für eine Menge $U$, ein Cover $C$ und eine natürliche Zahl $k$ prüft, ob $C$ ein Cover von $U$ ist.

Wir gehen davon aus, dass $U$ als Array und $C$ als Array von Arrays vorliegt. Zunächst prüfen wir, ob $|C| \le k$ gilt. Ist dies nicht der Fall, ist die Lösung ungültig. Falls dies gilt, muss anschließend geprüft werden, ob $C$ tatsächlich ein Cover von $U$ ist. Dazu iterieren wir über alle Arrays von $C$ und überschreiben jedes Element, dass wir in einem dieser Arrays finden, im Array $U$ mit \code{null}. Am Ende können wir über $U$ iterieren und prüfen, ob es Elemente ungleich \code{null} gibt. Falls ja, ist $C$ kein Cover von $U$. Andernfalls ist $C$ ein Cover von $U$ und die Lösung ist gültig.

Die Laufzeit des Algorithmus setzt sich wie folgt zusammen: Um $|C|$ zu ermitteln müssen wir einmal über das äußere Array von $C$ iterieren, was offensichtlich $\bO{|C|}$ kostet. Wenn der Rest ausgeführt muss $|C| \le k$ gelten. Außerdem wissen wir durch die Definition der Teilmenge, dass jedes Array in $C$ maximal $|U|$ Elemente enthalten kann. Also kostet das Iterieren über alle Elemente aus $C$ im worst case $\bO{k*|U|}$. Am Ende muss über $U$ iteriert werden, was offensichtlich $\bO{|U|}$ kostet. Die Gesamtlaufzeit beträgt also $\bO{|C| + k*|U| + |U|}$ und ist somit polynomiell.\\

Nun muss noch gezeigt werden, dass {\scshape SetCover} NP-schwer ist. Dazu zeigen wir, dass sich {\sc VertexCover} in polynomieller Zeit auf {\sc SetCover} reduzieren lässt. Dazu konstruieren wir zu einer gegebenen Instanz von {\sc VertexCover} in polynomieller Zeit eine Instanz von {\sc SetCover}.

Gegeben sei eine Instanz von {\sc VertexCover} mit einem Graphen $G=(V,E)$ und einer natürlichen Zahl $j$. Gesucht ist eine Instanz von {\sc SetCover} mit einem Universum $U$, einem Cover $C$ und einer natürlichen Zahl $k$. Für die Reduktion gelte $U = E$ und $k = j$. Weiterhin konstruieren wir $C$ so, dass wir jeden Knoten $v$ aus $G$ mit einem Label von 1 bis $n$ markieren, und für jeden Knoten $v_i$ eine Menge $S_i$ konstruieren, die genau aus den zu $v_i$ inzidenten Kanten besteht. Diese Mengen $S_i$ bilden dann genau $C$. Mit einer Adjazenzliste ist diese Konstruktion in $\bO{n+m} = \bO{n+m}$ möglich, da wir genau einmal über alle Knoten iterieren und dabei für jeden Knoten alle inzidenten Kanten (also im Prinzip alle Nachbarn) abfragen.

Nun muss lediglich gezeigt werden, dass genau dann wenn eine {\sc VertexCover}-Instanz gültig ist, auch die daraus konstruierbare {\sc SetCover}-Instanz gültig ist.

Gehen wir davon aus, dass eine gültige {\sc VertexCover}-Instanz existiert. D.h. es gibt genau $k$ Knoten, die alle Kanten des Graphen abdecken. Konstruiert man nun wie oben für jeden Knoten $v_i$ die dazugehörige Menge $S_i$, müssen diese Mengen offensichtlich ein Cover für das Universum $E$ sein. Für eine gültige {\sc VertexCover}-Instanz existiert also auch eine gültige {\sc SetCover}-Instanz.

Nun gehen wir davon aus, dass eine gültige {\sc SetCover}-Instanz existiert. Es existiert also $k$ Mengen $S_i$, die ein Cover für das Universum bilden. Dieses Universum sei nun die Kantenmenge eines Graphen. Nun lässt sich dieser Graph konstruieren, so dass für jede Menge $S_i$ ein Knoten $v_i$ existiert für den die Elemente aus $S_i$ die inzidenten Kanten zu $v_i$ darstellen. Offensichtlich bilden diese $k$ Knoten $v_i$ ein {\sc SetCover} des Graphen. Für eine gültige {\sc VertexCover}-Instanz existiert also auch eine gültige {\sc VertexCover}-Instanz.

Wir haben gezeigt, dass {\scshape SetCover} in NP liegt und dass {\scshape SetCover} NP-schwer ist. Somit ist {\scshape SetCover} auch NP-vollständig.

$\hfill \square$ 

\end{document}
