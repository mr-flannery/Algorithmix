\documentclass[12pt]{scrartcl}%{article} % Beginn der LaTeX-Datei

%% twocolumn

\usepackage{amsmath,amssymb}  % erleichtert Mathe 
\usepackage{enumerate}% schicke Nummerierung

\usepackage{graphicx} % für Grafik-Einbindung
%\usepackage{hyperref}

\usepackage[ngerman]{babel}
\usepackage[T1]{fontenc}
\usepackage{lmodern}
\usepackage[boxruled]{algorithm2e}
\usepackage{parskip}
 % Einstellungen, wenn man deutsch schreiben will, z.B. Trennregeln
\usepackage[utf8]{inputenc}  % für Unix-Systeme
  % ermöglicht die direkte Eingabe von Umlauten und ß
  % evt. obige Zeile ersetzen durch
  % \usepackage[ansinew]{inputenc}  % für Windows
  % \usepackage[applemac]{inputenc} % für den Mac
\usepackage{listings}

%%%%%%%%%%%%%%%%%%%%%%%%%%%%%%%%%%%%%%%%%%%%%%%%%%%%%%%%%%%%%%%%%%
%
%  ntheorem
%
\usepackage[thmmarks,amsmath,hyperref,noconfig]{ntheorem} 
  % erlaubt es, Sätze, Definitionen etc. einfach durchzunummerieren.
\newtheorem{satz}{Satz}[section] % Nummerierung nach Abschnitten
\newtheorem{hilfssatz}[satz]{Hilfssatz}
\newtheorem{kor}[satz]{Korollar}

\theorembodyfont{\upshape}
\newtheorem{beispiel}[satz]{Beispiel}
\newtheorem{bemerkung}[satz]{Bemerkung}
\newtheorem{definition}[satz]{Definition} %[section]

\theoremstyle{nonumberplain}
\theoremheaderfont{\itshape}
\theorembodyfont{\normalfont}
\theoremseparator{.}
\theoremsymbol{\ensuremath{_\blacksquare}}
\newtheorem{beweis}{Beweis}
\qedsymbol{\ensuremath{_\blacksquare}}
%\theoremclass{LaTeX}
%
% Ende ntheorem
%
%%%%%%%%%%%%%%%%%%%%%%%%%%%%%%%%%%%%%%%%%%%%%%%%%%%%%%%%%%%%%%%%%%


%\pagestyle{empty}
%
% Ändern der bedruckten Fläche der Seite
% \addtolength{\textwidth}{3cm}  % Befehl mit zwei Argumenten
% \addtolength{\textheight}{3cm}
% \hoffset-2cm % verschiebt das Textfenster nach links
% \voffset-5mm % verschiebt das Textfenster nach oben
%
\parindent=0pt %% keine Einzug zu Beginn von Abs\"atzen
\parskip=0.5mm   %% erzeugt einen zusätzliche Zeilenabstand zwischen
                %% Absätzen. Nötig bei \parindent=0pt


%%%%%%%%%%%%%%%%%%%%%%%%%%%%%%%%%%%%%%%%%%%%%%%%%%%%%%%%%%%%%%%%%%
% ermöglicht, farbigen Text zu drucken.
\usepackage{color}
% Und nun werden die Farben definiert - daran können Sie nach Belieben selber rumspielen.
\definecolor{white}{rgb}{1,1,1}
\definecolor{darkred}{rgb}{0.3,0,0}
\definecolor{darkgreen}{rgb}{0,0.3,0}
\definecolor{darkblue}{rgb}{0,0,0.3}
\definecolor{pink}{rgb}{0.78,0.09,0.51}
\definecolor{purple}{rgb}{0.28,0.24,0.55}
\definecolor{orange}{rgb}{1,0.6,0.0}
\definecolor{grey}{rgb}{0.4,0.4,0.4}
\definecolor{aquamarine}{rgb}{0.4,0.8,0.65}


\DeclareMathOperator{\GL}{GL} % einige Macro, siehe am Ende Abschn. 2
\newcommand{\N}{\mathbb{N}}
\newcommand{\Z}{\mathbb{Z}}
\newcommand{\Q}{\mathbb{Q}}
\newcommand{\R}{\mathbb{R}}
\newcommand{\C}{\mathbb{C}}
\newcommand{\cP}{{\mathcal P}}
\newcommand{\bO}[1]{\mathcal O(#1)}
\newcommand{\bT}[1]{\Theta (#1)}
\newcommand{\bOg}[1]{\Omega (#1)}

% Code-Macro, wenn man mal etwas als Code highlighten möchte
\newcommand{\code}[1]{\lstinline[basicstyle=\ttfamily\color{black}]{#1}}

\begin{document}

\author{Dennis Hempfing, Sebastian Koall}
\title{Übung 8}
\date{} 
\pagestyle{myheadings}
\markright{\hfill Dennis Hempfing, Sebastian Koall}

\maketitle % erzeugt den Kopf

\section*{Aufgabe 4}

\subsection*{Teilaufgabe a)}

Im worst case finden wir keinen Knoten mit Grad $> k$. In diesem Fall bleiben $n$ Knoten übrig. Wenn nun jeder Knoten einen Grad von $k$ hat, bleiben $\frac{n*k}{2}$ Kanten übrig (da eine Kante genau 2 Knoten verbinden).

\subsection*{Teilaufgabe b)}

Anhand der erläuterten Heurisitk (für eine gesuchtes {\sc VertexCover} der Größe $k$ muss entweder ein Knoten mit Grad $> k$ Teil des Covers sein oder alle seine Nachbarn) lässt sich ein \textit{bounded search tree algorithm} entwickeln, der entscheidet, ob ein gegebener Graph $G$ für ein gegebenes $k$ ein {\sc VertexCover} der Größe $k$ enthält.

\paragraph{Idee} Dieser Algorithmus arbeitet rekursiv. Zuerst werden die Knoten anhand ihres Grades sortiert. Nun nehmen wir den ersten Knoten mit Grad $> k$ (dieser Knoten sei im folgenden $v$ genannt) und tätigen zwei Aufrufe: zum einen wird \code{_vertexCover} auf $G\setminus v$ aufgerufen, zum anderen auf $G\setminus v.neighbors$ (jeweils mit k-1). Nun müssen folgende Fälle betrachtet werden: ist die Kantenmenge des Graphen leer, wird \code{true} zurückgegeben. Ist die Menge der Knoten im aktuellen Cover größer als $k$, wird \code{false} zurückgegeben. Ansonsten wird die Rückgabe der beiden Rekursionsaufrufe verodert und zurückgegeben.\\

\textbf{Anmerkung:} um den Pseudocode übersichtlich zu halten, wird auf eine gesonderte Angabe von Pseudocode für die Sortierung der Knoten verzichtet. Die Funktion \code{maxNode(n)} übernimmt die Sortierung der Knoten und gibt den ersten Knoten mit Grad $> k$ zurück. Sollte solch ein Knoten nicht vorhanden sein, wird \code{null} zurückgegeben.

\begin{algorithm}[ht]
		\LinesNumbered
		\caption{vertexCover(G, k)}
		\SetKwInOut{Input}{Eingabe}\SetKwInOut{Output}{Ausgabe}
		
		\Input{Graph G als Adjazenzliste, natürliche Zahl k}
		\Output{Wahrheitswert, G ein VertexCover der Größe k enthält}
		
		return \_vertexCover(G, k, k, 0)\;
\end{algorithm}

\begin{algorithm}[ht]
	\LinesNumbered
	\caption{\_vertexCover(G, k, curK, removedNodesCount)}
	\SetKwInOut{Input}{Eingabe}\SetKwInOut{Output}{Ausgabe}
	
	\Input{Graph G als Adjazenzliste, natürliche Zahl k, natürliche Zahl curK natürliche Zahl removedNodesCount}
	\Output{Wahrheitswert, G ein VertexCover der Größe k enthält}
	
	\If{G.edges.empty()} {\
		return true\;	
	}
	\If{removedNodesCount > k} {\
		return false\;
	}
	
	node = maxNode(curK)\;
	
	\If{node != null} {\
		return \_vertexCover((G - node), k, (curK - 1), (removedNodesCount + 1)) | \_vertexCover((G - node.neighbours), k, (curK - 1), (removedNodesCount + node.neighbours.size))\;
	} \Else {
		return \_vertexCover(G, k, (curK - 1), removedNodesCount)\;
	}
	
	
\end{algorithm}

\paragraph{Korrektheit}

Die Korrektheit des Algorithmus kann per Induktion bewiesen werden. 

Induktionsvoraussetzung ist in diesem Fall ein Graph mit leerer Kantenmenge. In diesem Fall gibt der Algorithmus \code{true} zurück, was korrekt ist.

Für den Induktionsschritt muss gezeigt werden, dass wenn der Algorithmus eine korrekte Antwort für $k$ liefert, dass er dann auch eine korrekte Antwort für $k+1$ liefert. Nehmen wir also an, dass das Ergebnis für $k$ korrekt ist. Dann gibt es für $k+1$ zwei Fälle. Im ersten Fall gibt es einen Knoten dessen Knotengrad größer als $k+1$ ist. Nach der Heurisitk muss also dieser oder alle seine Nachbarn teil des Covers sein. Da beide gefunden werden, wird auch ein korrektes Ergebnis geliefert. Im zweiten Fall gibt es keinen Knoten, dessen Grad größer als $k+1$ ist. Dann kann das Ergebnis entweder schon korrekt sein, oder nicht. Aber in jedem Fall wird es gefunden.

Damit ist die Korrektheit des Algorithmus gezeigt.

\paragraph{Laufzeit}

Grundsätzlich werden $k+1$ Fälle geprüft und im worst case gibt es zwei Rekursionsaufrufe pro Prüfung. Demzufolge wird ein Rekursionsbaum in $\bO{2^(k+1)}$ aufgebaut. Pro Rekursionsaufruf wird lediglich eine Operation ausgeführt, die nicht konstante Laufzeit benötigt, und das ist \code{maxNode(n)}. Aus der letzten Übung wissen wir, dass die Sortierung von Knoten nach ihrem Grad in $\bO{n}$ realisiert werden kann. Nun kann man argumentieren, dass es für jede Anzahl an Knoten $n$ einen Summanden $x$ gibt, so dass $2^kn \le 2^{k+x}$ und somit $2^kn \le 2^{\bO{k}}$ gilt. Somit hat der Algorithmus eine Gesamtlaufzeit von $\bO{2^{\bO{k}}}$, $\bO{2^{\bO{k}} + nk}$ stellt also eine gültige obere Schranke für den Algorithmus dar.

\end{document}
